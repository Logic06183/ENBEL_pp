\section{Conclusions}

This comprehensive analysis of climate-biomarker relationships in 11,398 clinical records from Johannesburg, South Africa, establishes that certain biomarkers---particularly hematological and lipid markers---show notable associations with meteorological conditions, while immune and inflammatory markers require alternative analytical approaches. The rigorous data integration and imputation methodologies developed here enable climate-socioeconomic-health research at scale and provide a reproducible framework for future investigations.

Our principal contributions include: (1) demonstration that 99.5\% climate coverage can be achieved for historical clinical cohorts through ERA5 reanalysis linkage, (2) development and validation of spatial-demographic imputation methods enabling integration of clinical and socioeconomic data, (3) identification of biomarkers with varying climate sensitivity using machine learning and explainable AI, and (4) transparent assessment of limitations including potential confounding and data leakage concerns.

The findings support incorporating meteorological context into biomarker interpretation, particularly for hematocrit and lipid panels, while emphasizing caution given confounding challenges. Public health strategies should address socioeconomic determinants of heat vulnerability alongside meteorological monitoring. Future research employing distributed lag non-linear models, within-person analyses, and experimental validation will strengthen causal inference and clarify mechanisms.

As climate change intensifies heat exposure globally, understanding how temperature affects human physiology becomes increasingly urgent. This work establishes biomarkers as quantifiable indicators of climate health effects and provides methodological tools for advancing this critical research agenda. We provide open-source analysis pipelines to facilitate replication and extension to diverse populations and settings.
