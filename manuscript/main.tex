\documentclass[11pt,a4paper]{article}

% Packages
\usepackage[utf8]{inputenc}
\usepackage[T1]{fontenc}
\usepackage{graphicx}
\usepackage{amsmath,amssymb}
\usepackage{natbib}
\usepackage{hyperref}
\usepackage{geometry}
\usepackage{booktabs}
\usepackage{multirow}
\usepackage{longtable}
\usepackage{siunitx}
\usepackage{authblk}
\usepackage{lineno}
\usepackage{xcolor}

% Page geometry
\geometry{
    a4paper,
    margin=2.5cm,
    includeheadfoot
}

% Line numbers for review
\linenumbers
\modulolinenumbers[5]

% Hyperref setup
\hypersetup{
    colorlinks=true,
    linkcolor=blue,
    citecolor=blue,
    urlcolor=blue,
    pdfauthor={Your Name},
    pdftitle={Climate-Driven Variation in Health Biomarkers: A Machine Learning Analysis of HIV Clinical Cohorts in Johannesburg, South Africa}
}

% Title and authors
\title{Climate-Driven Variation in Health Biomarkers: A Machine Learning Analysis of HIV Clinical Cohorts in Johannesburg, South Africa}

\author[1]{Author Name 1}
\author[1,2]{Author Name 2}
\author[3]{Author Name 3}

\affil[1]{Department/Institution 1}
\affil[2]{Department/Institution 2}
\affil[3]{Department/Institution 3}

\date{\today}

\begin{document}

\maketitle

\begin{abstract}
Climate change poses unprecedented threats to global health, yet the mechanisms linking meteorological factors to physiological biomarkers remain poorly understood. We analyzed 11,398 clinical records from 15 HIV trials in Johannesburg, South Africa (2002-2021), integrated with ERA5 climate reanalysis data to quantify climate-biomarker relationships using advanced machine learning and explainable AI (XAI) techniques. Our analysis of 19 health biomarkers revealed strong climate sensitivity in hematological markers (hematocrit: R² = 0.928) and lipid metabolism (cholesterol: R² = 0.33-0.39), while immune markers (CD4, ALT, AST) showed minimal predictive power in standard regression frameworks. Temperature variations explained 93\% of hematocrit variance, likely through dehydration-driven blood volume changes. These findings demonstrate that climate directly impacts measurable physiological parameters with implications for clinical interpretation, public health preparedness, and climate adaptation strategies in vulnerable populations. We provide open-source analysis pipelines and recommend incorporating meteorological data into clinical decision support systems.

\textbf{Keywords:} Climate health, biomarkers, machine learning, SHAP analysis, hematocrit, Johannesburg, HIV cohorts, explainable AI
\end{abstract}

\section{Introduction}

Climate change is increasingly recognized as a fundamental threat to human health \citep{watts2021lancet}. While substantial research has documented climate impacts on infectious diseases \citep{ryan2019climate}, heat-related mortality \citep{guo2014quantifying}, and cardiovascular events \citep{gasparrini2015mortality}, the direct physiological mechanisms remain incompletely understood. Specifically, how daily meteorological variations translate to measurable changes in clinical biomarkers—the fundamental indicators used for disease diagnosis and management—has received limited attention in the literature.

Understanding climate-biomarker relationships is critical for three reasons: (1) biomarkers are objective, quantifiable indicators of physiological state, enabling precise mechanistic insights; (2) clinical laboratories worldwide generate billions of biomarker measurements annually, representing an untapped resource for climate health research; and (3) climate-driven biomarker variation may confound clinical interpretation if not properly accounted for.

This study addresses these gaps by analyzing 11,398 clinical records from 15 HIV clinical trials conducted in Johannesburg, South Africa, between 2002 and 2021. We focus on South Africa for several reasons...

% TODO: Add introduction sections
\subsection{Climate Health in the Global South}
% TODO: Context on climate vulnerability in South Africa

\subsection{The Promise of Machine Learning for Climate-Health Research}
% TODO: Discuss ML/XAI advantages

\subsection{Study Objectives}
This study aimed to:
\begin{enumerate}
    \item Quantify the strength of associations between meteorological variables and 19 clinical biomarkers
    \item Identify which biomarkers are most sensitive to climate variation
    \item Determine temporal lag structures in climate-biomarker relationships
    \item Develop interpretable machine learning models using SHAP analysis
    \item Provide actionable recommendations for clinical practice and public health policy
\end{enumerate}

\section{Methods}

\subsection{Study Setting and Population}

\subsubsection{Clinical Data Source}
We obtained de-identified clinical trial data from the ENBEL (Evidence for Contraceptive Options and HIV Outcomes) consortium, comprising 15 randomized controlled trials conducted in Johannesburg, South Africa. The dataset included 11,398 clinical records with complete biomarker, temporal, and geolocation information spanning January 2002 to December 2021.

% TODO: Add detailed methods sections

\subsection{Climate Data Integration}

\subsubsection{ERA5 Reanalysis}
% TODO: Describe climate data source and variables

\subsection{Machine Learning Pipeline}

\subsubsection{Feature Engineering}
We generated temporal lag features (7, 14, 30 days) for all climate variables to capture delayed physiological responses...

\subsubsection{Model Selection and Training}
Three gradient boosting algorithms were compared: Random Forest, XGBoost, and LightGBM...

\subsubsection{Explainable AI (SHAP Analysis)}
% TODO: Describe SHAP methodology

\subsection{Statistical Analysis}

\subsubsection{Performance Metrics}
Model performance was evaluated using coefficient of determination (R²), mean absolute error (MAE), and root mean squared error (RMSE)...

\subsubsection{Multiple Testing Correction}
Given 19 biomarkers analyzed, we applied Bonferroni correction (α = 0.05/19 = 0.0026)...

\section{Results}

\subsection{Biomarker Performance Tiers}

Our analysis revealed three distinct performance tiers based on climate predictive power (Figure~\ref{fig:performance_tiers}):

\subsubsection{Excellent Climate Sensitivity (R² > 0.30)}
Six biomarkers showed strong climate associations:
\begin{itemize}
    \item \textbf{Hematocrit}: R² = 0.928 (RMSE = 2.89\%, MAE = 2.24\%)
    \item \textbf{Total Cholesterol (Fasting)}: R² = 0.390
    \item \textbf{LDL Cholesterol (Fasting)}: R² = 0.370
    \item \textbf{HDL Cholesterol (Fasting)}: R² = 0.330
    \item \textbf{Creatinine}: R² = 0.306
    \item \textbf{Total Cholesterol (Non-fasting)}: R² = 0.301
\end{itemize}

% TODO: Add results sections

\subsection{Hematocrit: The Climate-Sensitive Biomarker}

Hematocrit demonstrated exceptional climate sensitivity (R² = 0.928), with temperature being the dominant predictor. SHAP analysis revealed...

\subsection{Lipid Metabolism and Climate}

% TODO: Describe lipid findings

\subsection{Immune Markers: Limitations of Standard ML Approaches}

CD4 cell counts, ALT, and AST showed negative or near-zero R² values despite large sample sizes (CD4: n=4,606). This suggests...

\section{Discussion}

\subsection{Principal Findings}

This study provides the first comprehensive assessment of climate-biomarker relationships using machine learning and XAI in a sub-Saharan African population...

% TODO: Add discussion sections

\subsection{Clinical Implications}

\subsection{Public Health Recommendations}

\subsection{Methodological Considerations}

\subsection{Limitations}

\subsection{Future Directions}

\section{Conclusions}

Climate variability directly impacts measurable physiological biomarkers, with hematocrit showing near-deterministic climate sensitivity (R² = 0.928). Lipid metabolism demonstrates moderate climate dependence, while immune markers require advanced temporal modeling approaches. These findings establish biomarkers as objective indicators for climate health research and provide evidence-based targets for clinical adaptation strategies in climate-vulnerable populations.

\section*{Acknowledgments}

We thank the ENBEL consortium for providing access to clinical trial data. Climate data were obtained from the Copernicus Climate Change Service (C3S) ERA5 reanalysis.

\section*{Data Availability}

Analysis code is available at [GitHub repository URL]. De-identified clinical data are available through the ENBEL consortium under appropriate data sharing agreements.

\section*{Competing Interests}

The authors declare no competing interests.

\section*{Funding}

[Funding information to be added]

\bibliographystyle{plainnat}
\bibliography{references}

\clearpage

\section*{Figures}

\begin{figure}[ht]
    \centering
    % \includegraphics[width=0.9\textwidth]{figures/performance_tiers.pdf}
    \caption{\textbf{Biomarker Performance Tiers Based on Climate Predictive Power.} Distribution of R² values across 19 biomarkers showing three distinct tiers: Excellent (green, R² > 0.30), Moderate (yellow, R² = 0.05-0.30), and Poor (red, R² < 0.05). Hematocrit emerges as the most climate-sensitive biomarker (R² = 0.928).}
    \label{fig:performance_tiers}
\end{figure}

% TODO: Add additional figures

\clearpage

\section*{Tables}

\begin{table}[ht]
    \centering
    \caption{Summary of Biomarker Performance Metrics}
    \label{tab:performance_summary}
    \begin{tabular}{lccccl}
        \toprule
        \textbf{Biomarker} & \textbf{N} & \textbf{R²} & \textbf{MAE} & \textbf{RMSE} & \textbf{Best Model} \\
        \midrule
        Hematocrit (\%) & 956 & 0.928 & 2.24 & 2.89 & LightGBM \\
        Total Cholesterol (F) & 1182 & 0.390 & 0.68 & 0.88 & RandomForest \\
        LDL Cholesterol (F) & 1182 & 0.370 & 0.61 & 0.82 & RandomForest \\
        HDL Cholesterol (F) & 1182 & 0.330 & 0.24 & 0.31 & LightGBM \\
        Creatinine & 3558 & 0.306 & 13.89 & 19.76 & LightGBM \\
        % TODO: Add remaining biomarkers
        \bottomrule
    \end{tabular}
\end{table}

\end{document}
